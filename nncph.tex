\documentclass[a4paper,11pt]{article}
\usepackage[margin=1.75cm,paperheight=25.00cm,paperwidth=17.00cm]{geometry}
\usepackage{fontspec}
\usepackage{polyglossia}
\usepackage[style=nncph,backend=biber,url=true]{biblatex}
\addbibresource{bibliography.bib}
\usepackage[breaklinks=true,hidelinks]{hyperref}
\setmainlanguage{spanish}
\setmainfont{AltStandard}[RawFeature=lnum]
%
\begin{document}
\section*{\raggedleft\textsc{Novae normae ad citationem philologicam (NNCPh)}\\ \textsc{Nuevas normas para la citación filológica (NNCF)}}\par
%
Para libros con textos modernos = \textbf{Autor, A. (Fecha). \emph{Título} (Ed./Tr.= Nombre, N. y Nombre, N.). (Vol. X, T. Y/Fasc. Y, Z\textsuperscript{ed.}). Editorial. <Enlace>. (= \emph{Sigla}).}\par
Para libros con textos antiguos = \textbf{Editor, E. (Fecha). \emph{Título} (Au.= Nombre, N. \& Nombre, N.). (Vol. X, T. Y/Fasc. Y, Z\textsuperscript{ed.}). Editorial. <Enlace>. (= \emph{Sigla}).}\par
Con texto de revistas = \textbf{Autor, A. (Fecha). ``Título''. Revista, Nº, páginas. Editorial. <Enlace>.}\par
Para capítulos o artículos en libros o compilaciones = \textbf{Autor, A. (Fecha). ``Título''. \emph{Título del libro o compilación} (Serie o «colección», Nº), Nº págs. Editorial. <Enlace>.}\par
Con texto o material desde internet = \textbf{Autor, A. (Fecha/S. F.). ``Nombre''. <Enlace>. (= \emph{Sigla}).}\par
%
\section*{\textsc{Ejemplos}}
\textcite[113-120]{west2017} / \parencite[114-120]{west2017}.
\nocite{*}
\printbibliography[heading=none]
\end{document}